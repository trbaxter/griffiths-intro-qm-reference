\underline{\textbf{Practice Q6:}} \ \textbf{What is the average age?}

\bigskip

\underline{Solution}

Since \textit{j} represents an age, the average age can be written as
$\langle j \rangle$. In general:

\begin{align*}
    \langle j \rangle &= \dfrac{\sum \, j \cdot N(j)}{N} \\[1.5ex]
    &= \sum_{j=0}^{\infty} \, j \cdot P(j)
\end{align*}

\newpage

Using this approach:

\begin{align*}
    \dfrac{\dots \, + \, 14 \, + \, 15 \, + \, (3 \cdot 16) \, + \, \dots \,
    (2 \cdot 22) \, + \, (0 \cdot 23) \, + \, (2 \cdot 24) \, + \,
    (5 \cdot 25) \, + \, \dots}{14} &= \dfrac{294}{14}
    \, &= \, \boxed{21}
\end{align*}

\bigskip

We'll see this notation frequently throughout the textbook. Whenever ``expectation value'' is seen, interpret that as
``average value''. \textbf{Don't} mistake it with the \textit{most probable} value. These quantities are not the same!


Admittedly, this can make the terminology confusing to navigate. The only path forward is to to understand the
difference and be wary of the language being used.