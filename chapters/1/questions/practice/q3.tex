

\underline{\textbf{Practice Q3:}} \ \textbf{What's the total probability of
picking one person among all the people in the room?}

\bigskip

\underline{Solution}

It's likely obvious that the total probability is unity (we're
\textit{guaranteed} to get at least \underline{one} person) but let's prove
that with the following:

\[
    \sum_{j=0}^{\infty} \, P(j) = P(0) \, + \, P(1) \, + \dots \, + \, P(14) \,
    + \, P(15) \, + \, \dots \, + \, P(24) \, + \, P(25) \, + \, \dots
\]

Everything outside of the age range of 14--25 will have a contribution of 0,
so we can focus on these terms:

\begin{align*}
    &= \, P(14) \, + \, P(15) \, + \, P(16) \, + \, P(22) \, + \, P(24) \, + \,
    P(25) \\[1.5ex]
    &= \, \dfrac{1}{14} \, + \, \dfrac{1}{14} \, + \, \dfrac{3}{14} \, + \,
    \dfrac{2}{14} \, + \, \dfrac{2}{14} \, + \, \dfrac{5}{14} = \dfrac{14}{14}
    \, = \, \boxed{1}
\end{align*}