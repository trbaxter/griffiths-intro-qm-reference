We start off in Chapter 1 immediately stating that, to achieve an analogous
goal in quantum mechanics like solving F = ma for x(t) in classical physics,
we need to use the time-dependent Schrödinger equation, and utilize that with
an individual wave function representing a particle.

But we're not told \textit{why} we should even do that in the first place.

What's the justification for using a wave for something that,
at least classically, doesn't even remotely resemble something that's a wave?


Reason:

The only formalism that correctly predicts and explains QM experimental
outcomes is one that's wave-based.

- It's not a guess.

- A non-wave approach was tried many times and could not succeed.

- Necessity required a new formalism.

This does \textit{\textbf{not}} mean that the particle, itself, is a wave.
It means that we require a wave-based mathematical object
(the wavefunction $\Psi$) to understand the particle's behavior.
