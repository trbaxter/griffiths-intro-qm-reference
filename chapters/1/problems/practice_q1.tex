Imagine a room containing 14 people with the following age distribution:

Age 14 \ - \ 1

Age 15 \ - \ 1

Age 16 \ - \ 3

Age 22 \ - \ 2

Age 24 \ - \ 2

Age 25 \ - \ 5

\bigskip

Let $N(j)$ represent the number of people of age \textit{j} such that:

\[
    N = \sum_{j=0}^{\infty} \, N(j)
\]

Where N = 14, the total number of people. 

\bigskip \bigskip

\underline{\textbf{Practice Q1:}} \ \textbf{If you selected one individual at random from the room,
what is the probability that the person's age would be 15?}

\bigskip

\underline{Solution}

Recall the number of 15-year-olds and the total number of people in the distribution.

- There's one person aged 15

- There's 14 people total

\bigskip

The probability is given by:

\[
    P(j) = \dfrac{N(j)}{N} = \boxed{\dfrac{1}{14}}
\]

\bigskip \bigskip

\underline{\textbf{Practice Q2:}} \ \textbf{From the previous question, what is the probability of selecting someone
age 14 or someone age 15 from the room?}

\bigskip

\underline{Solution}

The probability of selecting someone who's 14 or 15 can be symbolically represented by the following:

\[
    P(14 \ \text{or} \ 15) = P(14 \cup 15)
\]

\newpage

Because this scenario is a mutually exclusive selection (someone can't be both
14 \textit{\textbf{\underline{and}}} 15 at the same time), this can be expressed
equivalently as:

\begin{align*}
    P(14 \, \text{or} \, 15) &= P(14) + P(15) \\[1.5ex]
    &= \dfrac{1}{14} + \dfrac{1}{14} \\[1.5ex]
    &= \dfrac{2}{14} \\[1.5ex]
    &= \boxed{\dfrac{1}{7}}
\end{align*}

Note that we took advantage of the following for this solution:

\[
    P(A \ \cup \ B) = P(A) + P(B) - P(A \ \cap \ B)
\]


Where, as we pointed out:

\[
    P(A \ \cap \ B) = 0
\]

\bigskip \bigskip

\underline{\textbf{Practice Q3:}} \ \textbf{What's the total probability of
picking one person among all the people in the room?}

\bigskip

\underline{Solution}

It's likely obvious that the total probability is unity (we're
\textit{guaranteed} to get at least \underline{one} person) but let's prove
that with the following:

\[
    \sum_{j=0}^{\infty} \, P(j) = P(0) \, + \, P(1) \, + \dots \, + \, P(14) \,
    + \, P(15) \, + \, \dots \, + \, P(24) \, + \, P(25) \, + \, \dots
\]

Everything outside of the age range of 14--25 will have a contribution of 0,
so we can focus on these terms:

\begin{align*}
    &= \, P(14) \, + \, P(15) \, + \, P(16) \, + \, P(22) \, + \, P(24) \, + \,
    P(25) \\[1.5ex]
    &= \, \dfrac{1}{14} \, + \, \dfrac{1}{14} \, + \, \dfrac{3}{14} \, + \,
    \dfrac{2}{14} \, + \, \dfrac{2}{14} \, + \, \dfrac{5}{14} = \dfrac{14}{14}
    \, = \, \boxed{1}
\end{align*}

\newpage

\underline{\textbf{Practice Q4:}} \ \textbf{What is the most probable age?}

\bigskip

\underline{Solution}

The most probable age is the age in which $P(j)$ is a maximum for all allowed
values of j.

\begin{align*}
    P(25) \, &= \, \dfrac{5}{14} \\[1.5ex]
    &\Rightarrow \, \boxed{\text{25 is the most probable age.}}
\end{align*}

\bigskip \bigskip

\underline{\textbf{Practice Q5:}} \ \textbf{What is the median age?}

\bigskip

\underline{Solution}

First off - ensure that the values are sorted from least to greatest.

For such a small dataset, it may be temping to cross-out values one-by-one on
both ends, but there's a superior way.

When the number of elements N is even, the median is the average of the elements
in the $\dfrac{N}{2}$ and $\dfrac{(N+1)}{2}$ positions. When the number of
elements is odd, the median is simply the element in the $\dfrac{N}{2}$
position.

\[
    \Rightarrow \dfrac{22+24}{2} = \boxed{23}
\]

\bigskip \bigskip

\underline{\textbf{Practice Q6:}} \ \textbf{What is the average age?}

\bigskip

\underline{Solution}

Since \textit{j} represents an age, the average age can be written as
$\langle j \rangle$. In general:

\begin{align*}
    \langle j \rangle &= \dfrac{\sum \, j \cdot N(j)}{N} \\[1.5ex]
    &= \sum_{j=0}^{\infty} \, j \cdot P(j)
\end{align*}

\newpage

Using this approach:

\begin{align*}
    \dfrac{\dots \, + \, 14 \, + \, 15 \, + \, (3 \cdot 16) \, + \, \dots \,
        (2 \cdot 22) \, + \, (0 \cdot 23) \, + \, (2 \cdot 24) \, + \,
        (5 \cdot 25) \, + \, \dots}{14} &= \dfrac{294}{14}
    \, &= \, \boxed{21}
\end{align*}

\bigskip

We'll see this notation frequently throughout the textbook. Whenever ``expectation value'' is seen, interpret that as
``average value''. \textbf{Don't} mistake it with the \textit{most probable} value. These quantities are not the same!


Admittedly, this can make the terminology confusing to navigate. The only path forward is to to understand the
difference and be wary of the language being used.

\bigskip \bigskip

\underline{\textbf{Practice Q7:}} \ \textbf{What is the average of the squares of the ages?}

\bigskip

\underline{Solution}

In this one, we make a slight modification to the average formula:

\begin{align*}
    \langle j^{2} \rangle &= \dfrac{\sum \, j^{2} \cdot N(j)}{N} \\[1.5ex]
    &= \sum_{j=0}^{\infty} \, j{^2} \cdot P(j)
\end{align*}

Using this:

\begin{align*}
    \dfrac{14^{2} \, + \, 15^{2} \, + \, (3 \cdot 16^{2}) \, + \, (2 \cdot 22^{2}) \, + \,
    + \, (2 \cdot 24^{2}) \, + \, (5 \cdot 25^{2})}{14} &= \dfrac{6434}{14} \\[1.5ex]
    &= \, 459.571 \\[1.5ex]
    &\approxeq \, \boxed{460}
\end{align*}

\bigskip

\textbf{*Beware:} $ \langle x^{2} \rangle \neq \langle x \rangle^{2}$ in general. Do not assume this works!

\bigskip \bigskip

\underline{\textbf{Practice Q7:}} \ \textbf{What is the average of the squares of the ages?}

\bigskip

\underline{Solution}