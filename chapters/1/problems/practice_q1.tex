Imagine a room containing 14 people with the following age distribution:

Age 14 \ - \ 1

Age 15 \ - \ 1

Age 16 \ - \ 3

Age 22 \ - \ 2

Age 24 \ - \ 2

Age 25 \ - \ 5

\bigskip

Let $N(j)$ represent the number of people of age \textit{j} such that:

\[
    N = \sum_{j=0}^{\infty} \, N(j)
\]

Where N = 14, the total number of people. 

\bigskip \bigskip

\underline{\textbf{Practice Q1:}} \ \textbf{If you selected one individual at
random from the room, what is the probability that the person's age would be
15?}

\bigskip

\underline{Solution}

Recall the number of 15-year-olds and the total number
of people in the distribution.

- There's one person aged 15

- There's 14 people total

\bigskip

The probability is given by:

\[
    P(j) = \dfrac{N(j)}{N} = \boxed{\dfrac{1}{14}}
\]

\bigskip \bigskip

\underline{\textbf{Practice Q2:}} \ \textbf{From the previous question, what is
the probability of selecting someone age 14 or someone age 15 from the room?}

\bigskip

\underline{Solution}

The probability of selecting someone who's 14 or 15 can be symbolically
represented by the following:

\[
    P(14 \ \text{or} \ 15) = P(14 \cup 15)
\]

\newpage

Because this scenario is a mutually exclusive selection (someone can't be both
14 \textit{\textbf{\underline{and}}} 15 at the same time), this can be expressed
equivalently as:

\begin{align*}
    P(14 \ \text{or} \ 15) &= P(14) + P(15) \\ \\
    &= \dfrac{1}{14} + \dfrac{1}{14} \\ \\
    &= \dfrac{2}{14} \\ \\
    &= \boxed{\dfrac{1}{7}}
\end{align*}

Note that we took advantage of the following for this solution:

\[
    P(A \ \cup \ B) = P(A) + P(B) - P(A \ \cap \ B)
\]


Where, as we pointed out:

\[
    P(A \ \cap \ B) = 0
\]

\bigskip \bigskip

\underline{\textbf{Practice Q3:}} \ \textbf{What's the total probability of
picking one person among all the people in the room?}

\bigskip

\underline{Solution}

It's likely obvious that the total probability is unity (we're
\textit{guaranteed} to get at least \underline{one} person) but let's prove
that with the following:

\[
    \sum_{j=0}^{\infty} \, P(j) = P(0) \, + \, P(1) \, + \dots \, + \, P(14) \,
    + \, P(15) \, + \, \dots \, + \, P(24) \, + \, P(25) \, + \, \dots
\]

Everything outside of the age range of 14--25 will have a contribution of 0,
so we can focus on these terms:

\begin{align*}
    &= \, P(14) \, + \, P(15) \, + \, P(16) \, + \, P(22) \, + \, P(24) \, + \,
    P(25) \\ \\
    &= \, \dfrac{1}{14} \, + \, \dfrac{1}{14} \, + \, \dfrac{3}{14} \, + \,
    \dfrac{2}{14} \, + \, \dfrac{2}{14} \, + \, \dfrac{5}{14} = \dfrac{14}{14}
    \, = \, \boxed{1}
\end{align*}

\newpage

\underline{\textbf{Practice Q4:}} \ \textbf{What is the most probable age?}

\bigskip

\underline{Solution}

The most probable age is the age in which $P(j)$ is a maximum for all allowed
values of j.

\begin{align*}
    P(25) \, &= \, \dfrac{5}{14} \\ \\
    &\Rightarrow \, \boxed{\text{25 is the most probable age.}}
\end{align*}