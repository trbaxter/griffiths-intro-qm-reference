Imagine a room containing 14 people with the following age distribution:

Age 14 \ - \ 1

Age 15 \ - \ 1

Age 16 \ - \ 3

Age 22 \ - \ 2

Age 24 \ - \ 2

Age 25 \ - \ 5

\bigskip

Let $N(j)$ represent the number of people of age \textit{j} such that:

\[
    N = \sum_{j=0}^{\infty} \, N(j)
\]

Where N = 14, the total number of people.

\bigskip \bigskip

\underline{\textbf{Practice Q1:}} \ \textbf{If you selected one individual at random from the room,
    what is the probability that the person's age would be 15?}

\bigskip

\underline{Solution}

Recall the number of 15-year-olds and the total number of people in the distribution.

- There's one person aged 15

- There's 14 people total

\bigskip

The probability is given by:

\[
    P(j) = \frac{N(j)}{N} = \boxed{\frac{1}{14}}
\]