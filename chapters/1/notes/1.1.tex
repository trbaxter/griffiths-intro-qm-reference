Classical mechanics approach to particles:

Determine \textit{x(t)} from $F = \dot{p}$ and initial conditions.

\bigskip \bigskip

Quantum approach:

Determine $\Psi(x,t)$ from $i\hbar \, \partial_{t}(\Psi) =
- \frac{\hbar^{2}}{2m} \, \partial_{x}^{2}(\Psi)
+ V\Psi$

\bigskip

This implies that the Schrödinger equation serves as something akin to
Newton's 2$^{\text{nd}}$ Law in a quantum mechanical context.

\bigskip \bigskip

Born's statistical interpretation of the wave function is given by the
following:

\[
    \int_{a}^{b} |\Psi|^{2} \, \text{d}x
\]

Which produces the probability of finding an object between points
\textit{a} and \textit{b} at time \textit{t}.

\bigskip \bigskip

Given the statistical nature of the wave function, there are different schools
of thought and interpretations of the subject. Two of the more prevailing
perspectives involve the Realist \& Orthodox positions.

\bigskip

\underline{Realist}
\begin{itemize}[leftmargin = 0.5in]
    \item QM is \textit{incomplete}.
    \item More information/variables are needed. The weirdness we experience
          is just a quirk.
\end{itemize}

\bigskip

\underline{Orthodox}
\begin{itemize}[leftmargin = 0.5in]
    \item QM \textit{is} complete.
    \item The weirdness of QM aren't quirks; they're \textbf{inherent} to
          reality.
\end{itemize}

\bigskip

Experimental results (at this point in time) have confirmed the Orthodox
position. Furthermore, any agnostic uncertainty is dispelled by discoveries
made by Bell's Theorem.

\newpage